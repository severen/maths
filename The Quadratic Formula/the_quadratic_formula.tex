\documentclass[parskip]{scrartcl}

%% Common Packages and Configuration

%% Layout and Style %%
\usepackage{enumitem}
\usepackage{booktabs}
\usepackage{xcolor}
\usepackage{hyperref}
\hypersetup{
  colorlinks=true,
  urlcolor=blue,
  pdfauthor={Severen Redwood},
}

%% Typography %%
\usepackage{fontspec}
\usepackage{microtype}

%% Maths %%
\usepackage{mathtools}
\usepackage{amssymb}
\usepackage{amsthm}
\usepackage[free-standing-units]{siunitx}
\usepackage{unicode-math}

% General
\DeclarePairedDelimiter{\card}{\lvert}{\rvert}
\DeclarePairedDelimiter{\abs}{\lvert}{\rvert}
\DeclarePairedDelimiter{\norm}{\lVert}{\rVert}
\DeclarePairedDelimiter{\ceil}{\lceil}{\rceil}
\DeclarePairedDelimiter{\floor}{\lfloor}{\rfloor}

\newcommand*{\conj}[1]{\overbar{#1}}

% Combinatorics
\newcommand*{\perm}[2]{\prescript{#1}{}P_{#2}}
\newcommand*{\comb}[2]{\prescript{#1}{}C_{#2}}

% Calculus
\DeclareMathOperator{\dif}{d\!}
\newcommand*{\od}[3][]{\frac{\dif{^{#1}}#2}{\dif{#3^{#1}}}}
\newcommand*{\pd}[3][]{\frac{\partial{^{#1}}#2}{\partial{#3^{#1}}}}

% Linear Algebra
\newcommand*{\uvec}[1]{\hat{#1}}
\newcommand*{\ihat}{\uvec{\imath}}
\newcommand*{\jhat}{\uvec{\jmath}}

% Hack in support for augmented matrices into the matrix commands.
\makeatletter
\renewcommand*{\env@matrix}[1][*\c@MaxMatrixCols c]{
  \hskip -\arraycolsep
  \let\@ifnextchar\new@ifnextchar
  \array{#1}
}
\makeatother

% Statistics
\newcommand*{\mean}[1]{\overbar{#1}}

% Theorem-style Environments
\newtheorem{theorem}{Theorem}
\newtheorem{proposition}{Proposition}
\newtheorem{result}{Result}
\theoremstyle{definition}
\newtheorem{definition}{Definition}

%% Language %%
\usepackage{csquotes}
\usepackage{polyglossia}
\setdefaultlanguage[variant=newzealand, ordinalmonthday]{english}


\usepackage{amsthm}
\newtheorem{theorem}{Theorem}

\title{The Quadratic Formula}
\author{Severen Redwood}

\begin{document}

\maketitle

\begin{abstract}
This paper will state and prove the quadratic formula.
\end{abstract}

\section{Introduction}

The quadratic formula is one of the most used and well-known formulae in
mathematics. Although the formula may seem rather obtuse and mystical to the
uninitiated, in reality it can be derived in a simple fashion via a
generalisation of the method of completing the square.

\section{The Quadratic Formula}

\begin{theorem}
The solutions of a quadratic equation of the form \(ax^2 + bx + c = 0\) can be
found with the formula

\[ x = \frac{-b ±\sqrt{b^2 - 4ac}}{2a}, \]

where \(a ≠ 0\).
\end{theorem}

\pagebreak

\begin{proof}
Starting with \(ax^2 + bx + c = 0\), complete the square:

\begin{align*}
    ax^2 + bx + c &= 0 \\
    ax^2 + bx &= -c \\
    x^2 + \frac{b}{a}x &= -\frac{c}{a} \\
    x^2 + \frac{b}{a}x + \frac{b^2}{4a^2} &= \frac{b^2}{4a^2} - \frac{c}{a} \\
    (x + \frac{b}{2a})^2 &= \frac{b^2}{4a^2} - \frac{c}{a}
\end{align*}

Then solve for \(x\):

\begin{align*}
    (x + \frac{b}{2a})^2 &= \frac{b^2}{2a^2} - \frac{c}{a} \\
    (x + \frac{b}{2a})^2 &= \frac{b^2}{4a^2} - \frac{4ac}{4a^2} \\
    (x + \frac{b}{2a})^2 &= \frac{b^2 - 4ac}{4a^2} \\
    x + \frac{b}{2a} &= ±\frac{\sqrt{b^2 - 4ac}}{2a} \\
    x &= -\frac{b}{2a} ±\frac{\sqrt{b^2 - 4ac}}{2a} \\
    x &= \frac{-b ± \sqrt{b^2 - 4ac}}{2a} \qedhere
\end{align*}
\end{proof}

\end{document}
