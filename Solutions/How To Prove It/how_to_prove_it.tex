\documentclass[headings=standardclasses]{scrartcl}

%% Common Packages and Configuration

% Generate ActualText annotations for better copy/paste in some PDF readers.
\XeTeXgenerateactualtext=1

% Style
\usepackage{xcolor}
\usepackage{hyperref}
\hypersetup{
  colorlinks=true,
  urlcolor=blue
}

% Maths
\usepackage{mathtools}
\usepackage{amssymb}
\usepackage{amsthm}
\newtheorem{theorem}{Theorem}
\usepackage{commath}
\usepackage{unicode-math}

% Language
\usepackage{csquotes}
\usepackage{polyglossia}
\setdefaultlanguage[variant=newzealand, ordinalmonthday]{english}

% Place horizontal rules under top-level section headings.
\makeatletter
\renewcommand{\sectionlinesformat}[4]{
  \ifstr{#1}{section}{
    \parbox[t]{\linewidth}{
      \@hangfrom{\hskip #2#3}{#4}\par
      \kern-0.9\ht\strutbox \rule{\linewidth}{0.6pt}
    }%
  }{\@hangfrom{\hskip #2#3}{#4}}
}
\makeatother


% Label lists with ([a..z]) in accordance with the book.
\setlist{label=(\alph*)}

% Theorem-style Environments
\newtheorem*{theorem}{Theorem}
\newtheorem*{conjecture}{Conjecture}
\theoremstyle{definition}
\newtheorem{exercise}{Exercise}[subsection]
\newenvironment{solution}{\begin{proof}[Solution]}{\end{proof}}

% Only display the subsection numbering for the exercise environment.
% e.g. 1.3.6 => 6
\renewcommand{\theexercise}{\arabic{exercise}}

% PDF Metadata
\hypersetup{
  pdftitle={Solutions to Exercises in How to Prove It (Third Edition)}
}

\title{Solutions to Exercises in\\ \emph{How to Prove It}\\ by Daniel J. Velleman}
\subtitle{Third Edition}
\author{Severen Redwood}
\date{}

\begin{document}

\maketitle

\tableofcontents

\section*{Introduction}
\addcontentsline{toc}{section}{Introduction}

\begin{exercise}~ % Very special space for creating a newline!
  \begin{enumerate}
    \item Factor \(2^{15} - 1 = 32,767\) into a product of two smaller positive
          integers.
    \item Find an integer \(x\) such that \(1 < x < 2^{32,767} - 1\) and
          \(2^{32,767} - 1\) is divisible by \(x\).
  \end{enumerate}
\end{exercise}

\begin{solution}
  The proof of Theorem 1 shows that \(2^n - 1 = xy\), where \(x ≔ 2^{b} - 1\)
  and \(y ≔ 1 + 2^b + 2^{2b} + ⋯ + 2^{(a - 1)b}\), and that \(2^{n} - 1 =
  2^{ab} - 1\), which implies \(n = ab\).

  \begin{enumerate}
    \item If \(n = 15 = 3 \cdot 5\), we can take \(a = 3\) and \(b = 5\).
          Therefore, \(x = 2^5 - 1 = 31\) and \(y = 1 + 2^5 + 2^{2 ⋅ 5} =
          1057\), and thus \(2^{15} - 1 = 31 ⋅ 1057 = 32,767\).
    \item If \(n = 32,767 = 7 \cdot 4,681\), we can take \(b = 7\). Therefore,
          \(x = 2^7 - 1 = 127\), which is clearly a divisor of \(2^{32,767} -
          1\) and necessarily less than \(2^{32,767} - 1\). \qedhere
  \end{enumerate}
\end{solution}

\begin{exercise}
  Make some conjectures about the values of \(n\) for which \(3^n − 1\) is
  prime or the values of \(n\) for which \(3^n − 2^n\) is prime.
\end{exercise}

\begin{solution}
  By looking at \autoref{table:conjecture_1}, we can make the following
  conjecture about \(3^n - 1\):

  \begin{conjecture}
    If \(n\) is an integer greater than or equal to \(1\), then \(3^n - 1\) is
    even.
  \end{conjecture}

  By looking at \autoref{table:conjecture_2}, we can make the following two
  conjectures about \(3^n - 2^n\):

  \begin{conjecture}
    If \(n\) is an integer greater than \(1\) and is prime, then \(3^n - 2^n\)
    is prime.
  \end{conjecture}

  \begin{conjecture}
    If \(n\) is an integer greater than \(1\) and is not prime, then \(3^n -
    2^n\) is not prime. \qedhere
  \end{conjecture}

  \begin{table}[H]
    \centering
    \begin{tabular}{rlrl}
      \toprule
      \(n\) & Is \(n\) prime? & \(3^n - 1\) & Is \(3^n - 1\) prime? \\
      \midrule
      1 & no & 2 & yes \\
      2 & yes & 8 & no \\
      3 & yes & 26 & no \\
      4 & no & 80 & no \\
      5 & yes & 242 & no \\
      6 & no & 728 & no \\
      7 & yes & 2186 & no \\
      8 & no & 6560 & no \\
      9 & no & 19682 & no \\
      10 & no & 59048 & no \\
      \bottomrule
    \end{tabular}
    \caption{Values of \(3^n - 1\).}\label{table:conjecture_1}
  \end{table}

  \begin{table}[H]
    \centering
    \begin{tabular}{rlrl}
      \toprule
      \(n\) & Is \(n\) prime? & \(3^n - 2^n\) & Is \(3^n - 2^n\) prime? \\
      \midrule
      1 & no & 1 & no \\
      2 & yes & 5 & yes \\
      3 & yes & 19 & yes \\
      4 & no & 65 & no \\
      5 & yes & 211 & yes \\
      6 & no & 665 & no \\
      7 & yes & 2059 & yes \\
      8 & no & 6305 & no \\
      9 & no & 19171 & no \\
      10 & no & 58025 & no \\
      \bottomrule
    \end{tabular}
    \caption{Values of \(3^n - 2^n\).}\label{table:conjecture_2}
  \end{table}
\end{solution}

\section{Sentential Logic}

\subsection{Deductive Reasoning and Logical Connectives}

\begin{exercise}
  Analyse the logical forms of the following statements:
  \begin{enumerate}
      \item We'll have either a reading assignment or homework problems, but we
            won't have both homework problems and a test.
      \item You won't go skiing, or you will and there won't be any snow.
      \item \(\sqrt{7} ≰  2\).
  \end{enumerate}
\end{exercise}

\begin{solution}
  The logical forms are:
  \begin{enumerate}
    \item \((R \lor H) \land \lnot (H \land T)\), where \(R\) stands for
          \enquote{we will have a reading assignment}, \(H\) for \enquote{we
          will have homework problems}, and \(T\) for \enquote{we will have a
          test}.
    \item \(\lnot G \lor (G \land \lnot S)\), where \(G\) stands for
          \enquote{you will go skiing} and \(S\) for \enquote{there will be
          snow}.
    \item \(\lnot ((\sqrt{7} < 2) \lor (\sqrt{7} = 2))\). \qedhere
  \end{enumerate}
\end{solution}

\begin{exercise}
  Analyse the logical forms of the following statements:
  \begin{enumerate}
    \item Either John and Bill are both telling the truth, or neither of them
          is.
    \item I'll have either fish or chicken, but I won't have both fish and
          mashed potatoes.
    \item \(3\) is a common divisor of \(6\), \(9\), and \(15\).
  \end{enumerate}
\end{exercise}

\begin{solution}
  The logical forms are:
  \begin{enumerate}
    \item \((J \land B) \lor \lnot (A \lor B)\), where \(J\) stands for
          \enquote{John is telling the truth} and \(B\) for \enquote{Bill is
          telling the truth}.
    \item \((F \lor C) \land \lnot (F \land M)\), where \(F\) stands for
          \enquote{I will have fish}, \(C\) for \enquote{I will have chicken},
          and \(M\) for \enquote{I will have mashed potatoes}.
    \item \(A \land B \land C\), where \(A\) stands for \enquote{3 divides 6},
          \(B\) for \enquote{3 divides 9}, and \(C\) for \enquote{3 divides
          15}. \qedhere
  \end{enumerate}
\end{solution}

\begin{exercise}
  Analyse the logical forms of the following statements:
  \begin{enumerate}
    \item Alice and Bob are not both in the room.
    \item Alice and Bob are both not in the room.
    \item Either Alice or Bob is not in the room.
    \item Neither Alice nor Bob is in the room.
  \end{enumerate}
\end{exercise}

\begin{solution}
  Let \(A\) stand for \enquote{Alice is in the room} and \(B\) for \enquote{Bob
  is in the room}. Then, the logical forms are:
  \begin{enumerate}
    \item \(\lnot (A \land B)\).
    \item \(\lnot A \land \lnot B\).
    \item \(\lnot A \lor \lnot B\).
    \item \(\lnot A \land \lnot B\). \qedhere
  \end{enumerate}
\end{solution}

\subsection{Truth Tables}

\begin{exercise}
  Make truth tables for the following formulae:
  \begin{enumerate}
    \item \(\lnot P \lor Q\).
    \item \((S \lor G) \land (\lnot S \lor \lnot G)\).
  \end{enumerate}
\end{exercise}

\begin{solution}
  The truth tables are:
  \begin{enumerate}
    \item
      \begin{tabular}[b]{ccc}
        \toprule
        \(P\) & \(Q\) & \(\lnot P \lor Q\) \\
        \midrule
        T & T & T \\
        T & F & F \\
        F & F & T \\
        F & T & T \\
        \bottomrule
      \end{tabular}
    \item
      \begin{tabular}[b]{ccc}
        \toprule
        \(S\) & \(G\) & \((S \lor G) \land (\lnot S \lor \lnot G)\) \\
        \midrule
        T & T & T \\
        T & F & F \\
        \bottomrule
      \end{tabular} \qedhere
  \end{enumerate}
\end{solution}

\section{Quantificational Logic}

\subsection{Quantifiers}

\begin{exercise}
  Analyse the logical forms of the following statements:
  \begin{enumerate}
    \item Anyone who has forgiven at least one person is a saint.
    \item Nobody in the calculus class is smarter than everybody in the
          discrete maths class.
    \item Everyone likes Mary, except Mary herself.
    \item Jane saw a police officer, and Roger saw one too.
    \item Jane saw a police officer, and Roger saw him too.
  \end{enumerate}
\end{exercise}

\begin{solution}
  The logical forms are:
  \begin{enumerate}
    \item Todo.
    \item Todo.
    \item Todo.
    \item Todo.
    \item Todo. \qedhere
  \end{enumerate}
\end{solution}

\section{Proofs}

\subsection{Proof Strategies}

\begin{exercise}
  Consider the following theorem. (This theorem was proved in the
  introduction.)

  \begin{theorem}
    Suppose \(n\) is an integer larger than \(1\) and \(n\) is not prime. Then
    \(2^n - 1\) is not prime.
  \end{theorem}

  \begin{enumerate}
    \item Identify the hypotheses and conclusion of the theorem. Are the
          hypotheses true when \(n = 6\)? What does the theorem tell you in
          this instance? Is it right?
    \item What can you conclude from the theorem in the case \(n = 15\)? Check
          directly that this conclusion is correct.
    \item What can you conclude from the theorem in the case \(n = 11\)?
  \end{enumerate}
\end{exercise}

\begin{solution}~
  \begin{enumerate}
    \item Todo.
    \item Todo.
    \item Todo. \qedhere
  \end{enumerate}
\end{solution}

\end{document}
