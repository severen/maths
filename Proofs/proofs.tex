\documentclass[headings=standardclasses]{scrartcl}

%% Common Packages and Configuration

%% Layout and Style %%
\usepackage{enumitem}
\usepackage{booktabs}
\usepackage{xcolor}
\usepackage{hyperref}
\hypersetup{
  colorlinks=true,
  urlcolor=blue,
  pdfauthor={Severen Redwood},
}

%% Typography %%
\usepackage{fontspec}
\usepackage{microtype}

%% Maths %%
\usepackage{mathtools}
\usepackage{amssymb}
\usepackage{amsthm}
\usepackage[free-standing-units]{siunitx}
\usepackage{unicode-math}

% General
\DeclarePairedDelimiter{\card}{\lvert}{\rvert}
\DeclarePairedDelimiter{\abs}{\lvert}{\rvert}
\DeclarePairedDelimiter{\norm}{\lVert}{\rVert}
\DeclarePairedDelimiter{\ceil}{\lceil}{\rceil}
\DeclarePairedDelimiter{\floor}{\lfloor}{\rfloor}

\newcommand*{\conj}[1]{\overbar{#1}}

% Combinatorics
\newcommand*{\perm}[2]{\prescript{#1}{}P_{#2}}
\newcommand*{\comb}[2]{\prescript{#1}{}C_{#2}}

% Calculus
\DeclareMathOperator{\dif}{d\!}
\newcommand*{\od}[3][]{\frac{\dif{^{#1}}#2}{\dif{#3^{#1}}}}
\newcommand*{\pd}[3][]{\frac{\partial{^{#1}}#2}{\partial{#3^{#1}}}}

% Linear Algebra
\newcommand*{\uvec}[1]{\hat{#1}}
\newcommand*{\ihat}{\uvec{\imath}}
\newcommand*{\jhat}{\uvec{\jmath}}

% Hack in support for augmented matrices into the matrix commands.
\makeatletter
\renewcommand*{\env@matrix}[1][*\c@MaxMatrixCols c]{
  \hskip -\arraycolsep
  \let\@ifnextchar\new@ifnextchar
  \array{#1}
}
\makeatother

% Statistics
\newcommand*{\mean}[1]{\overbar{#1}}

% Theorem-style Environments
\newtheorem{theorem}{Theorem}
\newtheorem{proposition}{Proposition}
\newtheorem{result}{Result}
\theoremstyle{definition}
\newtheorem{definition}{Definition}

%% Language %%
\usepackage{csquotes}
\usepackage{polyglossia}
\setdefaultlanguage[variant=newzealand, ordinalmonthday]{english}

% Place horizontal rules under top-level section headings.
\makeatletter
\renewcommand{\sectionlinesformat}[4]{
  \ifstr{#1}{section}{
    \parbox[t]{\linewidth}{
      \@hangfrom{\hskip #2#3}{#4}\par
      \kern-0.9\ht\strutbox \rule{\linewidth}{0.6pt}
    }%
  }{\@hangfrom{\hskip #2#3}{#4}}
}
\makeatother


\usepackage{xpatch}

\setmainfont{TeX Gyre Pagella}
\setmathfont{TeX Gyre Pagella Math}

% Reset theorem counters after unnumbered sections too.
\makeatletter
\xpretocmd{\@xthm}{\xpretocmd{\section}{\setcounter{#1}{0}}{}{}}{}{}
\makeatother

% Theorem-style Environments
\newtheorem{theorem}{Theorem}
\newtheorem{proposition}{Proposition}
\newtheorem{result}{Result}
\theoremstyle{definition}
\newtheorem{definition}{Definition}

% PDF Metadata
\hypersetup{
  pdftitle={Proof Playground}
}

\title{Proof Playground}
\author{Severen Redwood}
\date{}

\begin{document}

\maketitle

\noindent This document serves as my playground for practising the art of
writing mathematical proofs. As such, do not expect to find a proof of the
Riemann hypothesis here, or indeed anything else that is original. Instead, you
will find a range of proofs related to mostly standard undergraduate material
that I \textit{hope} are correct.

\section*{Algebra}

\begin{theorem}[Quadratic Formula]
  The solutions of the quadratic equation of the form \(ax^2 + bx + c = 0\) are
  \[ x = \frac{-b ±\sqrt{b^2 - 4ac}}{2a}. \]
\end{theorem}

\begin{proof}
  Let \(ax^{2} + bx + c = 0\). Complete the square to obtain
  \begin{equation*}
    {\left(x + \frac{b}{2a}\right)}^2 = \frac{b^{2}}{4a^{2}} - \frac{c}{a}.
  \end{equation*}
  Solve for \(x\) to obtain
  \begin{equation*}
    x = -\frac{b}{2a} ±\frac{\sqrt{b^2 - 4ac}}{2a}.
  \end{equation*}
  Thus,
  \begin{equation*}
    x = \frac{-b ± \sqrt{b^2 - 4ac}}{2a},
  \end{equation*}
  as desired.
\end{proof}

\begin{theorem}[Logarithm of a Product]
  \(\displaystyle \log_{b}(xy) = \log_{b}(x) + \log_{b}(y)\).
\end{theorem}

\begin{proof}
  Let \(x ≔ b^m\) and \(y ≔ b^n\). From the definition of a logarithm, it
  follows that \(\log_b(x) = m\) and \(\log_b(y) = n\).

  By substituting \(b^m\) and \(b^n\) for \(x\) and \(y\) in \(\log_b(xy)\),
  \begin{equation*}
  \begin{split}
    \log_b(xy) &= \log_b(b^{m}b^{n}) \\
               &= \log_b(b^{m + n}) \\
               &= m + n.
  \end{split}
  \end{equation*}
  Since \(m = \log_b(x)\) and \(n = \log_b(y)\), the above can be written as
  \begin{equation*}
    \log_b(xy) = \log_b(x) + \log_b(y),
  \end{equation*}
  which is the desired identity.
\end{proof}

\begin{theorem}[Logarithm of a Quotient]
  \(\displaystyle \log_{b}\left(\frac{x}{y}\right) = \log_{b}(x) -
  \log_{b}(y)\).
\end{theorem}

\begin{proof}
  Let \(x ≔ b^m\) and \(y ≔ b^n\). From the definition of a logarithm, it
  follows that \(\log_b(x) = m\) and \(\log_b(y) = n\).

  By substituting \(b^m\) and \(b^n\) for \(x\) and \(y\) in \(\log_b(x/y)\),
  \begin{equation*}
  \begin{split}
    \log_b\left(\frac{x}{y}\right) &= \log_b\left(\frac{b^m}{b^n}\right) \\
                                   &= \log_b(b^{m - n}) \\
                                   &= m - n.
  \end{split}
  \end{equation*}
  Since \(m = \log_b(x)\) and \(n = \log_b(y)\), the above can be written as
  \begin{equation*}
    \log_b\left(\frac{x}{y}\right) = \log_{b}(x) - \log_{b}(y),
  \end{equation*}
  which is the desired identity.
\end{proof}

\begin{theorem}[Logarithm of a Power]
  \(\displaystyle \log_{b}(x^y) = y\log_{b}(x)\).
\end{theorem}

\begin{proof}
  Let \(x ≔ b^n\). From the definition of a logarithm, it
  follows that \(\log_b(x) = n\).

  By substituting \(b^n\) for \(x\) in \(\log_b(x^y)\),
  \begin{equation*}
  \begin{split}
    \log_b(x^y) &= \log_b({(b^n)}^y) \\
                &= \log_b(b^{yn}) \\
                &= yn.
  \end{split}
  \end{equation*}
  Since \(n = \log_b(x)\), the above can be written as
  \begin{equation*}
    \log_b(x^y) = y\log_b(x),
  \end{equation*}
  which is the desired identity.
\end{proof}

\begin{theorem}[Change of Base Formula]
  Any logarithm can be rewritten in terms of another base with the formula
  \[ \log_{b}(a) = \frac{\log_{x}(a)}{\log_{x}(b)}. \]
\end{theorem}

\begin{proof}
  Let \(c ≔ \log_b(a)\). From the definition of a logarithm, it follows
  that \(b^c = a\). By taking the base-\(x\) logarithm of both sides,
  \begin{equation*}
      \log_x(b^c) = \log_x(a).
  \end{equation*}
  Therefore,
  \begin{align*}
      c\log_x(b) &= \log_x(a) \\
               c &= \frac{\log_x(a)}{\log_x(b)}.
  \end{align*}
  Since \(c = \log_b(a)\), the above can be written as
  \begin{equation*}
    \log_b(a) = \frac{\log_x(a)}{\log_x(b)},
  \end{equation*}
  which is the change of base formula.
\end{proof}

\section*{Number Theory}

\begin{definition}[Natural Number]
  A natural number is a member of the set \(ℕ = ℤ^{+} = \{1, 2, 3, \ldots\}\).
\end{definition}

\begin{result}
  The number \(\sqrt{2}\) is irrational.
\end{result}

\begin{proof}
  Suppose that \(\sqrt{2}\) is rational, and so it can be expressed as a ratio
  of two integers. Hence, let \(\sqrt{2} = a/b\), where \(a\) and \(b\) are
  coprime. By squaring both sides,
  \begin{equation*}
    2 = \frac{a^2}{b^2} ⟺ 2b^2 = a^2.
  \end{equation*}
  Therefore, \(a^2\) is even, and in turn, \(a\) is also even, since squares of
  odd integers are never even. Hence, there exists some integer \(k\) such that
  \(a = 2k\). By substituting this in,
  \begin{equation*}
    2b^2 = {(2k)}^2 ⟺ b^2 = 2k^2.
  \end{equation*}
  Using the same reasoning as with \(a\), it must be that \(b\) is also even.
  However, since \(a\) and \(b\) are both even, a contradiction arises. Two
  even integers cannot be coprime, so \(\sqrt{2}\) cannot be expressed as a
  ratio of two integers. Thus, \(\sqrt{2}\) is irrational.
\end{proof}

\begin{theorem}
  There are infinitely many prime numbers.
\end{theorem}

\begin{proof}
  Suppose there are only finitely many prime numbers. Let \(p_1, p_2, \ldots,
  p_n\) be a list of all primes and let \(m ≔ p_{1}p_{2} \cdots p_n + 1\). Note
  that \(m\) is not divisible by \(p_1\) since dividing \(m\) by \(p_1\) gives
  a remainder of \(1\). Similarly, \(m\) is not divisible by any other number
  in the list. Because \(m\) is larger than \(1\), \(m\) is either a prime or a
  product of primes.

  If \(m\) is a prime, then we have found a prime not in our list, which
  contradicts the assumption that it was a list of all prime numbers.

  If \(m\) is a product of primes, then it must be divisible by one of the
  primes in our list. However, we have shown \(m\) is not divisible by any
  number in the list. Thus the assumption that the list was a list of all prime
  numbers is again contradicted.

  Since the assumption that there are only finitely many prime numbers has led
  to a contradiction, there must be infinitely many prime numbers.
\end{proof}

\begin{result}
  If \(p\) and \(q\) are two consecutive primes that are each greater than
  \(2\), then \(p + q\) is a product of three integers that are each greater
  than \(1\).
\end{result}

\begin{proof}
  Without loss of generality, assume that \(p < q\). Note that \(p\) and \(q\)
  are both odd integers since they are both prime numbers greater than \(2\).
  Therefore, \(p + q = 2a\), for some integer \(a\). If \(a\) is prime, then
  \(p < a < q\) since \(a = (p + q)/2\). However, because \(p\) and \(q\) are
  consecutive primes, \(a\) cannot also be prime. Hence, \(a\) must be
  composite, and so the result follows.
\end{proof}

\begin{theorem}
  The sum of the first \(n\) natural numbers is equal to
  \[ \frac{n(n + 1)}{2}. \]
\end{theorem}

\begin{proof}
  We proceed by induction. If \(n = 1\), then the theorem is clearly true:
  \begin{equation*}
    ∑_{i = 1}^1 i = \frac{1(1 + 1)}{2} = 1.
  \end{equation*}
  So, the theorem holds for the base case of \(n = 1\).

  For the inductive hypothesis, assume the formula is true for all \(k > 1\).
  Hence,
  \begin{equation*}
    ∑_{i = 1}^k i = \frac{k(k + 1)}{2}.
  \end{equation*}

  For the inductive step, let \(n = k + 1\). By the properties of summation,
  \begin{equation*}
    ∑_{i = 1}^{k + 1} i = \sum_{i = 1}^{k} i + (k + 1).
  \end{equation*}
  By using the inductive hypothesis,
  \begin{equation*}
  \begin{split}
    ∑_{i = 1}^{k} i + (k + 1) &= \frac{k(k + 1)}{2} + (k + 1) \\
                              &= \frac{k(k + 1) + 2(k + 1)}{2} \\
                              &= \frac{(k + 1)(k + 2)}{2} \\
                              &= \frac{(k + 1)((k + 1) + 1)}{2}.
  \end{split}
  \end{equation*}
  So, the theorem holds when \(n = k + 1\).

  Since the base case and inductive step have been shown, the theorem holds for
  all natural numbers by the principle of mathematical induction.
\end{proof}

\begin{theorem}
  The sum of the squares of the first \(n\) natural numbers is equal to
  \[ \frac{n(n + 1)(2n + 1)}{6}. \]
\end{theorem}

\begin{proof}
  We proceed by induction. If \(n = 1\), then the theorem is clearly true:
  \begin{equation*}
    ∑_{i = 1}^1 i^2 = \frac{1(1 + 1)(2 ⋅1 + 1)}{6} = 1.
  \end{equation*}
  So, the theorem holds for the base case of \(n = 1\).

  For the inductive hypothesis, assume the formula is true for all \(k > 1\).
  Hence,
  \begin{equation*}
    ∑_{i = 1}^k i = \frac{k(k + 1)(2k + 1)}{6}.
  \end{equation*}

  For the inductive step, let \(n = k + 1\). By the properties of summation,
  \begin{equation*}
    ∑_{i = 1}^{k + 1} i^2 = ∑_{i = 1}^k i^2 + {(k + 1)}^2.
  \end{equation*}
  By using the inductive hypothesis,
  \begin{equation*}
  \begin{split}
    ∑_{i = 1}^k i^2 + {(k + 1)}^2 &= \frac{k(k + 1)(2k + 1)}{6} + (k + 1) \\
                                  &= \frac{k(k + 1)(2k + 1) + 6(k + 1)}{6} \\
                                  &= \frac{(k + 1)(k + 2)(2k + 3)}{6} \\
                                  &= \frac{(k + 1)((k + 1) + 1)(2(k + 1) + 1)}{6}.
  \end{split}
  \end{equation*}
  So, the theorem holds when \(n = k + 1\).

  Since the base case and inductive step have been shown, the theorem holds for
  all natural numbers by the principle of mathematical induction.
\end{proof}

\section*{Set Theory}

\begin{definition}[Ordered Pair]\label{def:ordered_pair}
  The ordered pair of two elements \(a\) and \(b\) is the set
  \[ (a, b) ≔ \{\{a\}, \{a, b\}\}. \]
\end{definition}

\begin{definition}[Cartesian Product]\label{def:cartesian_product}
  The Cartesian product of two sets \(A\) and \(B\) is the set
  \[ A × B ≔ \{\, (a, b) : a ∈ A, b ∈ B \,\}. \]
\end{definition}

\begin{theorem}
  For any sets \(A\), \(B\), and \(C\), the following hold:
  \begin{enumerate}[label=(\alph*)]
    \item \((A ∪ B) × C = (A × C) ∪ (B × C)\)
    \item \((A ∩ B) × C = (A × C) ∩ (B × C)\)
    \item \(A × (B ∪ C) = (A × B) ∪ (A × C)\)
    \item \(A × (B ∩ C) = (A × B) ∩ (A × C)\).
  \end{enumerate}
\end{theorem}

\begin{proof}[Proof of A]
  Let \((u, v) ∈ (A ∪ B) × C\). Therefore, \(u ∈ A ∪ B\) and \(v ∈ C\). This
  means that \(u ∈ A\) or \(u ∈ B\). If \(u ∈ A\), then \((u, v) ∈ A × C\). If
  \(u ∈ B\), then \((u, v) ∈ B × C\). Either way, \((u, v) ∈ (A × C) ∪ (B ×
  C)\). Hence,
  \begin{equation*}
    (A ∪ B) × C ⊆ (A × C) ∪ (B × C).
  \end{equation*}

  Now, let \(z ≔ (x, y) ∈ (A × C) ∪ (B × C)\). Either \(z ∈ A × C\) or \(z ∈ B
  × C\). In the first case, \(x ∈ A\) and \(y ∈ C\). In the second, \(x ∈ B\)
  and \(y ∈ C\), so \(z = (x, y) ∈ (A ∪ B) × C\). This implies that
  \begin{equation*}
    (A × C) ∪ (B × C) ⊆ (A ∪ B) × C.
  \end{equation*}

  Putting the two parts together completes the proof.
\end{proof}

\begin{proof}[Proof of B]
  Let \((u, v) ∈ (A ∩ B) × C\). Therefore, \(u ∈ A ∩ B\) and \(v ∈ C\). This
  means that \(u ∈ A\) and \(u ∈ B\). Thus, \((u, v) ∈ A × C\) and \((u, v) ∈ B
  × C\), and consequently, \((u, v) ∈ (A × C) ∩ (B × C)\). Hence,
  \begin{equation*}
    (A ∩ B) × C ⊆ (A × C) ∩ (B × C).
  \end{equation*}

  Now, let \(z ≔ (x, y) ∈ (A × C) ∩ (B × C)\).  Therefore, \(z ∈ A × C\) and
  \(z ∈ B × C\). So, \(x ∈ A\) and \(x ∈ B\), and likewise, \(y ∈ C\). Thus,
  \(z = (x, y) ∈ (A ∩ B) × C\). This implies that
  \begin{equation*}
       (A × C) ∩ (B × C) ⊆ (A ∩ B) × C.
  \end{equation*}

  Putting the two parts together completes the proof.
\end{proof}

\section*{Real Analysis}

\begin{definition}[Limit]
  Let \(f\) be a real-valued function defined on a subset \(D\) of the real
  numbers. Let \(c\) be a limit point of \(D\) and let \(L\) be a real number.
  We say that \[ \lim_{x → c} f(x) = L \] if for every \(ε > 0\), there exists
  a \(δ > 0\) such that, for all \(x ∈ D\), \[ 0 < \abs{x - c} < δ ⟹ \abs{f(x)
  - L} < ε. \]
\end{definition}

\begin{definition}[Partition of an Interval]
  A partition of a closed interval \([a, b]\) is a set of points \(P = \{x_0,
  x_1, \ldots, x_n\}\) satisfying the condition that \(a = x_0 < x_1 < \cdots <
  x_{n - 1} < x_n = b\).
\end{definition}

\begin{definition}[Norm of a Partition]
  The norm (or mesh) of a partition \(P = \{x_0, x_1, \ldots, x_n\}\) is the
  length of the longest subinterval of \(P\) and is denoted by \(\norm{P} =
  \max{\{ x_i - x_{i - 1} \}}_{i = 1}^n\).
\end{definition}

\begin{definition}[Riemann Sum]
  Let \(f\) be a function defined on a closed interval \(I\), and let \(P =
  \{x_0, x_1, \ldots, x_n\}\) be a partition of \(I\). A Riemann sum of \(f\)
  over \(I\) with partition \(P\) is defined as \[ ∑_{i = 1}^n f(x^*_i) Δx_i,
  \] where \(Δx_i = x_i - x_{i - 1}\) and \(x^*_i ∈ [x_{i - 1}, x_i]\).
\end{definition}

\begin{definition}[Definite Integral]
  Let \(f\) be a function defined on a closed interval \(I\), and let \(J\) be
  a real number. We say that \[ ∫_a^b f(x) \dif x = J, \] and that \(J\) is the
  limit of the Riemann sums \(∑_{i = 1}^n f(x^*_i) Δx_i\) if the following
  condition is satisfied:

  Given any \(ε > 0\), there is a corresponding \(ε > 0\) such that, for every
  partition \(P = \{x_0, x_1, \ldots, x_n\}\) of \(I\) with \(\norm{P} < δ\)
  and any sample point \(x^*_i ∈ [x_{i - 1}, x_i]\), we have that \[ \abs*{∑_{i
  = 1}^n f(x^*_i) Δx_i - J} < ε. \]
\end{definition}

\begin{result}
  \(\displaystyle \lim_{x → 2} (x^{2} + 1) = 5.\)
\end{result}

\begin{proof}
  Suppose \(ε > 0\). Let \(δ ≔ \min(1, ε/5)\) and \(x ∈ ℝ\) such that
  \(0 < \abs{x - 2} < δ\).

  Since \(\abs{x - 2} < δ\), it follows that
  \begin{equation*}
  \begin{split}
    \abs{x - 2} < 1 &⟹ -1 < x - 2 < 1 \\
                    &⟹ 1 < x < 3.
  \end{split}
  \end{equation*}
  In particular, this means that \(\abs{x + 2} < 5\). Likewise, it follows
  that \(\abs{x - 2} < ε/5\).

  Hence,
  \begin{equation*}
  \begin{split}
    \abs{x - 2} < δ &⟹ \abs{x - 2} < \frac{ε}{5} \\
                    &⟹ \abs{x + 2} \abs{x - 2} < 5 ⋅ \frac{ε}{5} \\
                    &⟹ \abs{x^{2} - 4} < ε \\
                    &⟹ \abs{(x^{2} + 1) - 5} < ε. \qedhere
  \end{split}
  \end{equation*}
\end{proof}

\begin{result}
  \(\displaystyle \lim_{x → 3} (x^{2} + 6) = 15\).
\end{result}

\begin{proof}
  Suppose \(ε > 0\). Let \(δ ≔ \min(1, ε/7)\) and \(x ∈ ℝ\) such that
  \(0 < \abs{x - 3} < δ\).

  Since \(\abs{x - 3} < δ\), it follows that
  \begin{equation*}
  \begin{split}
    \abs{x - 3} < 1 &⟹ -1 < x - 3 < 1 \\
                    &⟹ 2 < x < 4.
  \end{split}
  \end{equation*}
  In particular, this means that \(\abs{x + 3} < 7\). Likewise, it follows
  that \(\abs{x - 3} < ε/7\).

  Hence,
  \begin{equation*}
  \begin{split}
    \abs{x - 3} < δ &⟹ \abs{x - 3} < \frac{ε}{7} \\
                    &⟹ \abs{x + 3} \abs{x - 3} < 7 ⋅ \frac{ε}{7} \\
                    &⟹ \abs{x^{2} - 9} < ε \\
                    &⟹ \abs{(x^{2} + 6) - 15} < ε. \qedhere
  \end{split}
  \end{equation*}
\end{proof}

\begin{result}
  \(\displaystyle \lim_{x → 0} \frac{x}{x^{2} + 1} = 0\).
\end{result}

\begin{proof}
  Suppose \(ε > 0\). Let \(δ ≔ \min(1, 2ε)\) and \(x ∈ ℝ\) such that
  \(0 < \abs{x} < δ\).

  Since \(\abs{x} < δ\), it follows that \(\abs{x} < 1\), and thus
  \(\abs{x^{2} + 1} < 2\). Likewise, it follows that \(\abs{x} < 2ε\).

  Hence,
  \begin{equation*}
  \begin{split}
    \abs{x} < 2ε &⟹ \frac{\abs{x}}{\abs{x^{2} + 1}} < \frac{2ε}{2} \\
                 &⟹ \frac{\abs{x}}{\abs{x^{2} + 1}} < ε \\
                 &⟹ \abs*{\frac{x}{x^{2} + 1}} < ε. \qedhere
  \end{split}
  \end{equation*}
\end{proof}

\begin{result}
  \(\displaystyle \lim_{x → 5^{+}} \frac{1}{x - 5} = ∞\).
\end{result}

\begin{proof}
  Suppose \(M > 0\). Let \(δ ≔ 1/M\) and \(x ∈ ℝ\) such that
  \(0 < x - 5 < δ\).

  Since \(x - 5 < δ\), it follows that
  \begin{equation*}
    x - 5 < \frac{1}{M} ⟹ \frac{1}{x - 5} > M. \qedhere
  \end{equation*}
\end{proof}

\begin{theorem}[Constant Multiple of Definite Integrals]
  If \(f\) is a integrable function, then \[ ∫_a^b kf(x) \dif x = k∫_a^b f(x)
  \dif x, \] for some constant \(k\).
\end{theorem}

\begin{proof}
  From the definition of a definite integral, we have that
  \begin{equation*}
      ∫_a^b kf(x) \dif x = \lim_{\norm{P} → 0} ∑_{i = 1}^n kf(x^*_i) Δx_i.
  \end{equation*}
  By the properties of limits and summation,
  \begin{equation*}
  \begin{split}
      \lim_{\norm{P} → 0} ∑_{i = 1}^n kf(x^*_i) Δx_i &= k\lim_{\norm{P} → 0} ∑_{i = 1}^n f(x^*_i) Δx_i \\
      &= k∫_a^b f(x) \dif x,
  \end{split}
  \end{equation*}
  as desired.
\end{proof}

\end{document}
