\documentclass[parskip]{scrartcl}

%% Common Packages and Configuration

%% Layout and Style %%
\usepackage{enumitem}
\usepackage{booktabs}
\usepackage{xcolor}
\usepackage{hyperref}
\hypersetup{
  colorlinks=true,
  urlcolor=blue,
  pdfauthor={Severen Redwood},
}

%% Typography %%
\usepackage{fontspec}
\usepackage{microtype}

%% Maths %%
\usepackage{mathtools}
\usepackage{amssymb}
\usepackage{amsthm}
\usepackage[free-standing-units]{siunitx}
\usepackage{unicode-math}

% General
\DeclarePairedDelimiter{\card}{\lvert}{\rvert}
\DeclarePairedDelimiter{\abs}{\lvert}{\rvert}
\DeclarePairedDelimiter{\norm}{\lVert}{\rVert}
\DeclarePairedDelimiter{\ceil}{\lceil}{\rceil}
\DeclarePairedDelimiter{\floor}{\lfloor}{\rfloor}

\newcommand*{\conj}[1]{\overbar{#1}}

% Combinatorics
\newcommand*{\perm}[2]{\prescript{#1}{}P_{#2}}
\newcommand*{\comb}[2]{\prescript{#1}{}C_{#2}}

% Calculus
\DeclareMathOperator{\dif}{d\!}
\newcommand*{\od}[3][]{\frac{\dif{^{#1}}#2}{\dif{#3^{#1}}}}
\newcommand*{\pd}[3][]{\frac{\partial{^{#1}}#2}{\partial{#3^{#1}}}}

% Linear Algebra
\newcommand*{\uvec}[1]{\hat{#1}}
\newcommand*{\ihat}{\uvec{\imath}}
\newcommand*{\jhat}{\uvec{\jmath}}

% Hack in support for augmented matrices into the matrix commands.
\makeatletter
\renewcommand*{\env@matrix}[1][*\c@MaxMatrixCols c]{
  \hskip -\arraycolsep
  \let\@ifnextchar\new@ifnextchar
  \array{#1}
}
\makeatother

% Statistics
\newcommand*{\mean}[1]{\overbar{#1}}

% Theorem-style Environments
\newtheorem{theorem}{Theorem}
\newtheorem{proposition}{Proposition}
\newtheorem{result}{Result}
\theoremstyle{definition}
\newtheorem{definition}{Definition}

%% Language %%
\usepackage{csquotes}
\usepackage{polyglossia}
\setdefaultlanguage[variant=newzealand, ordinalmonthday]{english}


\newtheorem*{theorem}{Theorem}

\title{Proofs}
\author{Severen Redwood}

\begin{document}

\maketitle

\section{About This Document}

This document consists of various assorted proofs that I have written down, both
for fun and in order to enhance my knowledge and understanding. Nothing in here
is novel or new, so do not expect to find an amazing proof of the Riemann
hypothesis, the Collatz conjecture, or anything else like that.

\section{The Quadratic Formula}

\begin{theorem}
  The solutions of a quadratic equation of the form \(ax^{2} + bx + c = 0\) are
  \[ x = \frac{-b ±\sqrt{b^{2} - 4ac}}{2a}. \]
\end{theorem}

\begin{proof}
  Let \(ax^{2} + bx + c = 0\). Completing the square gives the equation
  \begin{align*}
    ax^{2} + bx + c &= 0 \\
    ax^{2} + bx &= -c \\
    x^{2} + \frac{b}{a}x &= -\frac{c}{a} \\
    x^{2} + \frac{b}{a}x + \frac{b^{2}}{4a^{2}} &= \frac{b^{2}}{4a^{2}} - \frac{c}{a} \\
    {(x + \frac{b}{2a})}^{2} &= \frac{b^{2}}{4a^{2}} - \frac{c}{a}.
  \end{align*}

  Solving for \(x\) results in
  \begin{align*}
    {(x + \frac{b}{2a})}^{2} &= \frac{b^{2}}{2a^{2}} - \frac{c}{a} \\
    {(x + \frac{b}{2a})}^{2} &= \frac{b^{2}}{4a^{2}} - \frac{4ac}{4a^{2}} \\
    {(x + \frac{b}{2a})}^{2} &= \frac{b^{2} - 4ac}{4a^{2}} \\
    x + \frac{b}{2a} &= ±\frac{\sqrt{b^{2} - 4ac}}{2a} \\
    x &= -\frac{b}{2a} ±\frac{\sqrt{b^{2} - 4ac}}{2a} \\
    x &= \frac{-b ± \sqrt{b^{2} - 4ac}}{2a},
  \end{align*}
  which is the quadratic formula.
\end{proof}

\section{Irrationality of \(\sqrt{2}\)}

\begin{theorem}
  The square root of two (\(\sqrt{2}\)) is irrational.
\end{theorem}

\begin{proof}
  Suppose that \(\sqrt{2}\) is rational and thus can be written as the ratio of
  two integers \(\frac{a}{b}\), where \(a\) is prime to \(b\) (\(a ⟂ b\)).
  Therefore, we can state that
  \begin{align*}
    \sqrt{2} &= \frac{a}{b} \\
    2 &= \frac{a^{2}}{b^{2}}.
  \end{align*}

  Rearranging for \(a^{2}\) gives the equation
  \begin{equation*}
    a^{2} = 2b^{2},
  \end{equation*}
  which shows that \(a^{2}\) is even, since \(2b^{2}\) is necessarily even
  because it is a multiple of 2.

  It follows that \(a\) must also be even, as squares of odd integers are never
  even. Therefore, there exists some integer \(k\) that satisfies the equation
  \(a = 2k\).

  Substituting \(2k\) for \(a\) leads to
  \begin{align*}
    {(2k)}^{2} &= 2b^{2} \\
    4k^{2} &= 2b^{2} \\
    2k^{2} &= b^{2},
  \end{align*}
  which demonstrates that \(b^{2}\) and therefore \(b\) itself must be even,
  using the same reasoning as with \(a^{2}\).

  Since \(a\) and \(b\) are both even, the assumption that \(\sqrt{2}\) is
  rational and \(a\) is prime to \(b\) no longer holds, as they share a common
  factor of 2. Therefore, \(\sqrt{2}\) must be \textbf{irrational}.
\end{proof}

\section{Logarithmic Identities}

\subsection{The Product Law}

\begin{theorem}
  A logarithm of the form \(\log_{b}(xy)\) is equivalent to \(\log_{b}(x) +
  \log_{b}(y)\).
\end{theorem}

\begin{proof}
  Let \(x = b^{m}\) and \(y = b^{n}\). From the definition of a logarithm, it
  follows that \(\log_{b}(x) = m\) and \(\log_{b}(y) = n\).

  Substituting \(b^{m}\) and \(b^{n}\) for \(x\) and \(y\) in the expression
  \(\log_{b}(xy)\) gives the equation
  \begin{equation*}
    \begin{split}
      \log_{b}(xy) &= \log_{b}(b^{m} · b^{n}) \\
      &= \log_{b}(b^{m + n}) \\
      &= m + n.
    \end{split}
  \end{equation*}

  Since \(m = \log_{b}(x)\) and \(n = \log_{b}(y)\), the above can be written as
  \begin{equation*}
    \log_{b}(xy) = \log_{b}(x) + \log_{b}(y),
  \end{equation*}
  which is the desired identity.
\end{proof}

\subsection{The Quotient Law}

\begin{theorem}
  A logarithm of the form \(\log_{b}(\frac{x}{y})\) is equivalent to
  \(\log_{b}(x) - \log_{b}(y)\).
\end{theorem}

\begin{proof}
  Let \(x = b^{m}\) and \(y = b^{n}\). From the definition of a logarithm, it
  follows that \(\log_{b}(x) = m\) and \(\log_{b}(y) = n\).

  Substituting \(b^{m}\) and \(b^{n}\) for \(x\) and \(y\) in the expression
  \(\log_{b}(\frac{x}{y})\) gives the equation
  \begin{equation*}
    \begin{split}
      \log_{b}\big(\frac{x}{y}\big) &= \log_{b}\big(\frac{b^{m}}{b^{n}}\big) \\
      &= \log_{b}(b^{m - n}) \\
      &= m - n.
    \end{split}
  \end{equation*}

  Since \(m = \log_{b}(x)\) and \(n = \log_{b}(y)\), the above can be written as
  \begin{equation*}
    \log_{b}\big(\frac{x}{y}\big) = \log_{b}(x) - \log_{b}(y)
  \end{equation*}
  which is the desired formula.
\end{proof}

\subsection{The Power Law}

\begin{theorem}
  A logarithm of the form \(\log_{b}(x^{y})\) is equivalent to \(y\log_{b}(x)\).
\end{theorem}

\begin{proof}
  Let \(x = b^{n}\). From the definition of a logarithm, it follows that
  \(\log_{b}(x) = n\).

  Substituting \(b^{n}\) for \(x\) in the expression \(\log_{b}(x^{y})\) gives
  the equation
  \begin{equation*}
    \begin{split}
      \log_{b}(x^{y}) &= \log_{b}({(b^{n})}^{y}) \\
      &= \log_{b}(b^{ny}) \\
      &= ny
    \end{split}
  \end{equation*}

  Since \(n = \log_{b}(x)\), the above can be written as
  \begin{equation*}
    \begin{split}
      \log_{b}(x^{y}) &= \log_{b}(x) · y \\
      &= y\log_{b}(x)
    \end{split}
  \end{equation*}
  which is the desired formula.
\end{proof}

\subsection{The Change of Base Formula}

\begin{theorem}
  Any logarithm \(\log_{b}(a)\) can be rewritten in terms of another base with
  the formula
  \[ \log_{b}(a) = \frac{\log_{x}(a)}{\log_{x}(b)}. \]
\end{theorem}

\begin{proof}
  Let \(c = \log_{b}(a)\). From the definition of a logarithm, it follows that
  \(b^c = a\).

  Taking the base-\(x\) logarithm on both sides gives the equation
  \begin{equation*}
    \log_{x}(b^c) = \log_{x}(a).
  \end{equation*}

  Applying the logarithmic power law and solving for \(c\) leads to the equation
  \begin{equation*}
    \begin{split}
      c\log_{x}(b) &= \log_{x}(a) \\
      c &= \frac{\log_{x}(a)}{\log_{x}(b)}.
    \end{split}
  \end{equation*}

  Since \(c = \log_{b}(a)\), the above can be written as
  \begin{equation*}
    \log_{b}(a) = \frac{\log_{x}(a)}{\log_{x}(b)}
  \end{equation*}
  which is the change of base formula.
\end{proof}

\end{document}
