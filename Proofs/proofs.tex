\documentclass[parskip]{scrartcl}

%% Common Packages and Configuration

% Generate ActualText annotations for better copy/paste in some PDF readers.
\XeTeXgenerateactualtext=1

% Style
\usepackage{xcolor}
\usepackage{hyperref}
\hypersetup{
  colorlinks=true,
  urlcolor=blue
}

% Maths
\usepackage{mathtools}
\usepackage{amssymb}
\usepackage{amsthm}
\newtheorem{theorem}{Theorem}
\usepackage{commath}
\usepackage{unicode-math}

% Language
\usepackage{csquotes}
\usepackage{polyglossia}
\setdefaultlanguage[variant=newzealand, ordinalmonthday]{english}


\usepackage{amsthm}
\newtheorem{theorem}{Theorem}

\title{Proofs}
\author{Severen Redwood}

\begin{document}

\maketitle

\section{About This Document}

This document contains various assorted proofs that I have written down in order
to assist my understanding. Nothing in here is novel or new, these are all
simply retellings of that which has already been proven.

\section{The Quadratic Formula}

\begin{theorem}
  The solutions of a quadratic equation of the form \(ax^2 + bx + c = 0\) can be
  found with the formula

  \[ x = \frac{-b ±\sqrt{b^2 - 4ac}}{2a}, \]

  where \(a ≠ 0\).
\end{theorem}

\begin{proof}
  Completing the square on the general quadratic equation \(ax^2 + bx + c = 0\)
  gives

  \begin{align*}
    ax^2 + bx + c &= 0 \\
    ax^2 + bx &= -c \\
    x^2 + \frac{b}{a}x &= -\frac{c}{a} \\
    x^2 + \frac{b}{a}x + \frac{b^2}{4a^2} &= \frac{b^2}{4a^2} - \frac{c}{a} \\
    {(x + \frac{b}{2a})}^2 &= \frac{b^2}{4a^2} - \frac{c}{a}.
  \end{align*}

  This allows for isolating \(x\), giving

  \begin{align*}
    {(x + \frac{b}{2a})}^2 &= \frac{b^2}{2a^2} - \frac{c}{a} \\
    {(x + \frac{b}{2a})}^2 &= \frac{b^2}{4a^2} - \frac{4ac}{4a^2} \\
    {(x + \frac{b}{2a})}^2 &= \frac{b^2 - 4ac}{4a^2} \\
    x + \frac{b}{2a} &= ±\frac{\sqrt{b^2 - 4ac}}{2a} \\
    x &= -\frac{b}{2a} ±\frac{\sqrt{b^2 - 4ac}}{2a} \\
    x &= \frac{-b ± \sqrt{b^2 - 4ac}}{2a},
  \end{align*}

  which is the quadratic formula.
\end{proof}

\section{Irrationality of \(\sqrt{2}\)}

\begin{theorem}
  The square root of two, \(\sqrt{2}\), can not be represented as a ratio of two
  integers, and therefore is irrational.
\end{theorem}

\begin{proof}
  Suppose that \(\sqrt{2}\) is rational and thus can be written as the ratio of
  two integers \(\frac{a}{b}\), where \(a\) is prime to \(b\) (\(a ⟂
  b\)). Therefore, we can state that

  \begin{align*}
    \sqrt{2} &= \frac{a}{b} \\
    2 &= \frac{a^2}{b^2}.
  \end{align*}

  Rearranging for \(a^2\) gives the equation

  \begin{equation*}
    a^2 = 2b^2,
  \end{equation*}

  which shows that \(a^2\) is even, since \(2b^2\) is necessarily even because
  it is a multiple of 2.

  It follows that \(a\) must also be even, as squares of odd integers are never
  even. Therefore, there exists some integer \(k\) that satisfies the equation
  \(a = 2k\).

  Substituting \(2k\) for \(a\) leads to

  \begin{align*}
    {(2k)}^2 &= 2b^2 \\
    4k^2 &= 2b^2 \\
    2k^2 &= b^2,
  \end{align*}

  which demonstrates that \(b^2\) and therefore \(b\) itself must be even, using
  the same reasoning as with \(a^2\).

  Since \(a\) and \(b\) are both even, the assumption that \(\sqrt{2}\) is
  rational and \(a\) is prime to \(b\) no longer holds, as they share a common
  factor of 2.
\end{proof}

\section{Logarithms}

\subsection{The Product Law}

\begin{theorem}
  A logarithm of the form \(\log_b(xy)\) can be expressed as \(\log_b(x) +
  \log_b(y)\).
\end{theorem}

\begin{proof}
  Let \(x = b^m\) and \(y = b^n\). From the definition of a logarithm, it
  follows that \(\log_b(x) = m\) and \(\log_b(y) = n\).

  Substituting \(b^m\) and \(b^n\) for \(x\) and \(y\) in the expression
  \(\log_b(xy)\) gives the equation

  \begin{equation*}
    \begin{split}
      \log_b(xy) &= \log_b(b^m · b^n) \\
      &= \log_b(b^{m + n}) \\
      &= m + n.
    \end{split}
  \end{equation*}

  Since \(m = \log_b(x)\) and \(n = \log_b(y)\), the above can be written as

  \begin{equation*}
    \log_b(xy) = \log_b(x) + \log_b(y)
  \end{equation*}

  with the use of substitution.
\end{proof}

\subsection{The Quotient Law}

\begin{theorem}
  A logarithm of the form \(\log_b(\frac{x}{y})\) can be expressed as
  \(\log_b(x) - \log_b(y)\).
\end{theorem}

\begin{proof}
  Let \(x = b^m\) and \(y = b^n\). From the definition of a logarithm, it
  follows that \(\log_b(x) = m\) and \(\log_b(y) = n\).

  Substituting \(b^m\) and \(b^n\) for \(x\) and \(y\) in the expression
  \(\log_b(\frac{x}{y})\) gives the equation

  \begin{equation*}
    \begin{split}
      \log_b\big(\frac{x}{y}\big) &= \log_b\big(\frac{b^m}{b^n}\big) \\
      &= \log_b(b^{m - n}) \\
      &= m - n.
    \end{split}
  \end{equation*}

  Since \(m = \log_b(x)\) and \(n = \log_b(y)\), the above can be written as

  \begin{equation*}
    \log_b\big(\frac{x}{y}\big) = \log_b(x) - \log_b(y)
  \end{equation*}

  with the use of substitution.
\end{proof}

\subsection{The Power Law}

\begin{theorem}
  A logarithm of the form \(\log_b(x^y)\) can be expressed as \(y ·
  \log_b(x)\).
\end{theorem}

\begin{proof}
  Let \(x = b^{c}\). From the definition of a logarithm, it follows that
  \(\log_b(x) = c\).

  Substituting \(b^c\) for \(x\) in the expression \(\log_b(x^y)\) gives the
  equation

  \begin{equation*}
    \begin{split}
      \log_b(x^y) &= \log_b({(b^c)}^y) \\
      &= \log_b(b^{cy}) \\
      &= cy
    \end{split}
  \end{equation*}

  Since \(x = \log_b\), the above can be written as

  \begin{equation*}
    \begin{split}
      \log_b(x^y) &= \log_b(x) · y \\
      &= y · \log_b(x)
    \end{split}
  \end{equation*}

  with the use of substitution.
\end{proof}

\subsection{The Change of Base Formula}

\begin{theorem}
  Any logarithm \(\log_b(a)\) can be re-expressed in terms of another base with the formula

  \[ \log_b(a) = \frac{\log_x(a)}{\log_x(b)}. \]
\end{theorem}

\begin{proof}

  Let \(c = \log_b(a)\). From the definition of a logarithm, it follows that
  \(b^c = a\).

  Taking the base-\(x\) logarithm on both sides gives the equation

  \begin{equation*}
    \log_x(b^c) = \log_x(a).
  \end{equation*}

  Applying the logarithmic power law and solving for \(c\) leads to the equation

  \begin{equation*}
    \begin{split}
      c\log_x(b) &= \log_x(a) \\
      c &= \frac{\log_x(a)}{\log_x(b)}.
    \end{split}
  \end{equation*}

  Since \(c = \log_b(a)\), the above can be written as

  \begin{equation*}
    \log_b(a) = \frac{\log_x(a)}{\log_x(b)}
  \end{equation*}

  with the use of substitution.
\end{proof}

\end{document}
