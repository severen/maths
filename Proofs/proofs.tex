\documentclass[parskip]{scrartcl}

%% Common Packages and Configuration

%% Layout and Style %%
\usepackage{enumitem}
\usepackage{booktabs}
\usepackage{xcolor}
\usepackage{hyperref}
\hypersetup{
  colorlinks=true,
  urlcolor=blue,
  pdfauthor={Severen Redwood},
}

%% Typography %%
\usepackage{fontspec}
\usepackage{microtype}

%% Maths %%
\usepackage{mathtools}
\usepackage{amssymb}
\usepackage{amsthm}
\usepackage[free-standing-units]{siunitx}
\usepackage{unicode-math}

% General
\DeclarePairedDelimiter{\card}{\lvert}{\rvert}
\DeclarePairedDelimiter{\abs}{\lvert}{\rvert}
\DeclarePairedDelimiter{\norm}{\lVert}{\rVert}
\DeclarePairedDelimiter{\ceil}{\lceil}{\rceil}
\DeclarePairedDelimiter{\floor}{\lfloor}{\rfloor}

\newcommand*{\conj}[1]{\overbar{#1}}

% Combinatorics
\newcommand*{\perm}[2]{\prescript{#1}{}P_{#2}}
\newcommand*{\comb}[2]{\prescript{#1}{}C_{#2}}

% Calculus
\DeclareMathOperator{\dif}{d\!}
\newcommand*{\od}[3][]{\frac{\dif{^{#1}}#2}{\dif{#3^{#1}}}}
\newcommand*{\pd}[3][]{\frac{\partial{^{#1}}#2}{\partial{#3^{#1}}}}

% Linear Algebra
\newcommand*{\uvec}[1]{\hat{#1}}
\newcommand*{\ihat}{\uvec{\imath}}
\newcommand*{\jhat}{\uvec{\jmath}}

% Hack in support for augmented matrices into the matrix commands.
\makeatletter
\renewcommand*{\env@matrix}[1][*\c@MaxMatrixCols c]{
  \hskip -\arraycolsep
  \let\@ifnextchar\new@ifnextchar
  \array{#1}
}
\makeatother

% Statistics
\newcommand*{\mean}[1]{\overbar{#1}}

% Theorem-style Environments
\newtheorem{theorem}{Theorem}
\newtheorem{proposition}{Proposition}
\newtheorem{result}{Result}
\theoremstyle{definition}
\newtheorem{definition}{Definition}

%% Language %%
\usepackage{csquotes}
\usepackage{polyglossia}
\setdefaultlanguage[variant=newzealand, ordinalmonthday]{english}


\usepackage{amsthm}
\newtheorem{theorem}{Theorem}

\title{Proofs}
\author{Severen Redwood}

\begin{document}

\maketitle

\section{About This Document}

This document contains various assorted proofs that I have written down in order
to assist my understanding. Nothing in here is novel or new, these are all
simply retellings of that which has already been proven.

\section{The Quadratic Formula}

\begin{theorem}
The solutions of a quadratic equation of the form \(ax^2 + bx + c = 0\) can be
found with the formula

\[ x = \frac{-b ±\sqrt{b^2 - 4ac}}{2a}, \]

where \(a ≠ 0\).
\end{theorem}

\begin{proof}
Completing the square on the general quadratic equation \(ax^2 + bx + c = 0\) gives

\begin{align*}
    ax^2 + bx + c &= 0 \\
    ax^2 + bx &= -c \\
    x^2 + \frac{b}{a}x &= -\frac{c}{a} \\
    x^2 + \frac{b}{a}x + \frac{b^2}{4a^2} &= \frac{b^2}{4a^2} - \frac{c}{a} \\
    {(x + \frac{b}{2a})}^2 &= \frac{b^2}{4a^2} - \frac{c}{a}.
\end{align*}

This allows for isolating \(x\), giving

\begin{align*}
    {(x + \frac{b}{2a})}^2 &= \frac{b^2}{2a^2} - \frac{c}{a} \\
    {(x + \frac{b}{2a})}^2 &= \frac{b^2}{4a^2} - \frac{4ac}{4a^2} \\
    {(x + \frac{b}{2a})}^2 &= \frac{b^2 - 4ac}{4a^2} \\
    x + \frac{b}{2a} &= ±\frac{\sqrt{b^2 - 4ac}}{2a} \\
    x &= -\frac{b}{2a} ±\frac{\sqrt{b^2 - 4ac}}{2a} \\
    x &= \frac{-b ± \sqrt{b^2 - 4ac}}{2a},
\end{align*}

which is the quadratic formula.
\end{proof}

\section{Irrationality of \(\sqrt{2}\)}

\begin{theorem}
The square root of two, \(\sqrt{2}\), can not be represented as a ratio of two
integers, and therefore is irrational.
\end{theorem}

\begin{proof}
Suppose that \(\sqrt{2}\) is rational and thus can be written as the ratio of
two integers \(\frac{a}{b}\), where \(a\) is prime to \(b\) (\(a ⟂
b\)). Therefore, we can state that

\begin{align*}
  \sqrt{2} &= \frac{a}{b} \\
  2 &= \frac{a^2}{b^2}.
\end{align*}

Rearranging for \(a^2\) gives the equation

\begin{equation*}
  a^2 = 2b^2,
\end{equation*}

which shows that \(a^2\) is even, since \(2b^2\) is necessarily even because it
is a multiple of 2.

It follows that \(a\) must also be even, as squares of odd integers are never
even. Therefore, there exists some integer \(k\) that satisfies the equation \(a
= 2k\).

Substituting \(2k\) for \(a\) leads to

\begin{align*}
  {(2k)}^2 &= 2b^2 \\
  4k^2 &= 2b^2 \\
  2k^2 &= b^2,
\end{align*}

which demonstrates that \(b^2\) and therefore \(b\) itself must be even, using
the same reasoning as with \(a^2\).

Because \(a\) and \(b\) are both even, the assumption that \(\sqrt{2}\) is
rational and \(a\) is prime to \(b\) no longer holds as they share a common
factor of 2.
\end{proof}

\end{document}
